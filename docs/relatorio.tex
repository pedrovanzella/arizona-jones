\documentclass[12pt]{article}
\usepackage[T1]{fontenc}
\usepackage{algorithm2e}
\usepackage{dot2texi}

\usepackage{tikz}
\usetikzlibrary{shapes,arrows}

\makeatletter
%%%%%%%%%%%%%%%%%%%%%%%%%%%%%% Textclass specific LaTeX commands.
\newenvironment{lyxlist}[1]
{\begin{list}{}
{\settowidth{\labelwidth}{#1}
 \setlength{\leftmargin}{\labelwidth}
 \addtolength{\leftmargin}{\labelsep}
 \renewcommand{\makelabel}[1]{##1\hfil}}}
{\end{list}}

%%%%%%%%%%%%%%%%%%%%%%%%%%%%%% User specified LaTeX commands.
\usepackage{sbc-template}

\usepackage[brazil]{babel}
\usepackage[utf8]{inputenc}

\sloppy

\title{Arizona Jones}

\author{Pedro Vanzella\inst{1}}

\address{Faculdade de Informática -- Pontifícia Universidade Católica do Rio
Grande do Sul (PUCRS) \\ Av. Ipiranga, 6681 - Porto Alegre / RS / Brasil
    \email pedro@pedrovanzella.com}

\makeatother

\usepackage{babel}
\usepackage{listings}
\lstset {
    mathescape,
    frame=none
}
\renewcommand{\lstlistingname}{Listagem}

\begin{document}

\maketitle
\begin{abstract}
  A solution for the Arizona Jones in the Tel Dor Temple is proposed.
\end{abstract}
\begin{resumo}
  Uma solu\c{c}ão para o problema do Arizona Jones no Templo de Tel Dor é proposta.
\end{resumo}


\section{Introdução}\label{sec:intro}

O problema a ser resolvido consiste em encontrar, dentro de um conjunto de números, a maior seqüência crescente de números que, em base 6, têm somente um dígito de diferen\c{c}a entre um e outro.

Por exemplo, no conjunto $782$ $229$ $446$ $527$ $874$ $19$ $829$ $70$ $830$ $992$ $430$ $649$, a maior seqüência é $649$ $829$ $830$.

\section{Estruturas de Dados}\label{sec:estruturas}
Vamos representar o conjunto de números como um grafo. Isto nos permitirá ligar os nodos que formam seqüências válidas e caminhar pelo grafo procurando o maior caminho entre todos.

Para representar internamente o grafo, utilizamos três dicionários: um dicionário para nodos, um para vértices e um para o peso dos nodos. Este último será explicado em detalhe na se\c{c}ão~\ref{sec:algorimos:achar-maior-caminho}.

O dicionário de nodos tem uma propriedade interessante: seu índice é o número em base 10, como foi lido do arquivo; seu conteúdo é a representa\c{c}ão em base 6 do mesmo número, como um array de dígitos, para facilitar a compara\c{c}ão.

% TODO: EXPLICAR ISTO!
%O grafo é dirigido e acíclico, então o algoritmo de encontrar o maior caminho é O(n): https://en.wikipedia.org/wiki/Longest_path_problem

Podemos ver na Figura~\ref{fig:testeprof} um grafo gerado de acordo com o exemplo dado na se\c{c}ão~\ref{sec:intro}

\begin{figure}[h!]
\begin{dot2tex}[neato,options=-tmath]
  \documentclass{article}
\usepackage[x11names, rgb]{xcolor}
\usepackage[utf8]{inputenc}
\usepackage{tikz}
\usetikzlibrary{snakes,arrows,shapes}
\usepackage{amsmath}
%
%

%

%

\begin{document}
\pagestyle{empty}
%
%
%

\enlargethispage{100cm}
% Start of code
% \begin{tikzpicture}[anchor=mid,>=latex',line join=bevel,]
\begin{tikzpicture}[>=latex',line join=bevel,]
  \pgfsetlinewidth{1bp}
%%
\pgfsetcolor{black}
  % Edge: 829 -- 830
  \draw [] (459.0bp,71.697bp) .. controls (459.0bp,60.846bp) and (459.0bp,46.917bp)  .. (459.0bp,36.104bp);
  % Edge: 649 -- 829
  \draw [] (459.0bp,143.7bp) .. controls (459.0bp,132.85bp) and (459.0bp,118.92bp)  .. (459.0bp,108.1bp);
  % Node: 830
\begin{scope}
  \definecolor{strokecol}{rgb}{0.0,0.0,0.0};
  \pgfsetstrokecolor{strokecol}
  \draw (459.0bp,18.0bp) ellipse (27.0bp and 18.0bp);
  \draw (459.0bp,18.0bp) node {830};
\end{scope}
  % Node: 782
\begin{scope}
  \definecolor{strokecol}{rgb}{0.0,0.0,0.0};
  \pgfsetstrokecolor{strokecol}
  \draw (27.0bp,162.0bp) ellipse (27.0bp and 18.0bp);
  \draw (27.0bp,162.0bp) node {782};
\end{scope}
  % Node: 992
\begin{scope}
  \definecolor{strokecol}{rgb}{0.0,0.0,0.0};
  \pgfsetstrokecolor{strokecol}
  \draw (603.0bp,162.0bp) ellipse (27.0bp and 18.0bp);
  \draw (603.0bp,162.0bp) node {992};
\end{scope}
  % Node: 19
\begin{scope}
  \definecolor{strokecol}{rgb}{0.0,0.0,0.0};
  \pgfsetstrokecolor{strokecol}
  \draw (387.0bp,162.0bp) ellipse (27.0bp and 18.0bp);
  \draw (387.0bp,162.0bp) node {19};
\end{scope}
  % Node: 874
\begin{scope}
  \definecolor{strokecol}{rgb}{0.0,0.0,0.0};
  \pgfsetstrokecolor{strokecol}
  \draw (315.0bp,162.0bp) ellipse (27.0bp and 18.0bp);
  \draw (315.0bp,162.0bp) node {874};
\end{scope}
  % Node: 446
\begin{scope}
  \definecolor{strokecol}{rgb}{0.0,0.0,0.0};
  \pgfsetstrokecolor{strokecol}
  \draw (171.0bp,162.0bp) ellipse (27.0bp and 18.0bp);
  \draw (171.0bp,162.0bp) node {446};
\end{scope}
  % Node: 527
\begin{scope}
  \definecolor{strokecol}{rgb}{0.0,0.0,0.0};
  \pgfsetstrokecolor{strokecol}
  \draw (243.0bp,162.0bp) ellipse (27.0bp and 18.0bp);
  \draw (243.0bp,162.0bp) node {527};
\end{scope}
  % Node: 829
\begin{scope}
  \definecolor{strokecol}{rgb}{0.0,0.0,0.0};
  \pgfsetstrokecolor{strokecol}
  \draw (459.0bp,90.0bp) ellipse (27.0bp and 18.0bp);
  \draw (459.0bp,90.0bp) node {829};
\end{scope}
  % Node: 229
\begin{scope}
  \definecolor{strokecol}{rgb}{0.0,0.0,0.0};
  \pgfsetstrokecolor{strokecol}
  \draw (99.0bp,162.0bp) ellipse (27.0bp and 18.0bp);
  \draw (99.0bp,162.0bp) node {229};
\end{scope}
  % Node: 70
\begin{scope}
  \definecolor{strokecol}{rgb}{0.0,0.0,0.0};
  \pgfsetstrokecolor{strokecol}
  \draw (531.0bp,162.0bp) ellipse (27.0bp and 18.0bp);
  \draw (531.0bp,162.0bp) node {70};
\end{scope}
  % Node: 649
\begin{scope}
  \definecolor{strokecol}{rgb}{0.0,0.0,0.0};
  \pgfsetstrokecolor{strokecol}
  \draw (459.0bp,162.0bp) ellipse (27.0bp and 18.0bp);
  \draw (459.0bp,162.0bp) node {649};
\end{scope}
  % Node: 430
\begin{scope}
  \definecolor{strokecol}{rgb}{0.0,0.0,0.0};
  \pgfsetstrokecolor{strokecol}
  \draw (675.0bp,162.0bp) ellipse (27.0bp and 18.0bp);
  \draw (675.0bp,162.0bp) node {430};
\end{scope}
%
\end{tikzpicture}
% End of code

%
\end{document}
%




\end{dot2tex}
  \label{fig:testeprof}
  \caption{Exemplo de grafo}
  \end{figure}


\section{Algoritmos}\label{sec:algoritmos}


\subsection{Achar Maior Caminho}\label{sec:algoritmos:achar-maior-caminho}

\begin{lstlisting}
escadinha(base):
  count $\leftarrow$ 0
  num $\leftarrow$ [0] 
	
  enquanto num $\neq$ maxnum: 
    num $\leftarrow$ incrementa(num, base) 
    se valido(num):
      count++

  $\rightarrow$ count
\end{lstlisting}

Onde {\sf [0]} é um {\em array} contendo o inteiro $0$.



\section{Resultados}\label{sec:resultados}
Obviamente o algoritmo é ruim. Mas quão ruim? Seria possível resolver o problema para uma entrada razoavel? A fim de curiosidade, o algorítmo foi implementado e os resultados podem ser vistos na Tabela~\ref{tab:resultados-1}.

\begin{table}[h]
\caption{Primeira Tentativa}
\label{tab:resultados-1}
\begin{tabular}{ll}
  {\sf num} & Tempo (s) \\
  \hline
  2 & 0.01 \\
  3 & 0.01 \\
  4 & 0.01 \\
  5 & 0.02 \\
  6 & 0.09 \\
  7 & 1.25 \\
  8 & 23.9 \\
  9 & 540.9    
\end{tabular}
\end{table}



\end{document}
