\documentclass[12pt]{article}
\usepackage[T1]{fontenc}
\usepackage{algorithm2e}

\makeatletter
%%%%%%%%%%%%%%%%%%%%%%%%%%%%%% Textclass specific LaTeX commands.
\newenvironment{lyxlist}[1]
{\begin{list}{}
{\settowidth{\labelwidth}{#1}
 \setlength{\leftmargin}{\labelwidth}
 \addtolength{\leftmargin}{\labelsep}
 \renewcommand{\makelabel}[1]{##1\hfil}}}
{\end{list}}

%%%%%%%%%%%%%%%%%%%%%%%%%%%%%% User specified LaTeX commands.
\usepackage{sbc-template}

\usepackage[brazil]{babel}
\usepackage[utf8]{inputenc}

\sloppy

\title{Arizona Jones}

\author{Pedro Vanzella\inst{1}}

\address{Faculdade de Informática -- Pontifícia Universidade Católica do Rio
Grande do Sul (PUCRS) \\ Av. Ipiranga, 6681 - Porto Alegre / RS / Brasil
    \email pedro@pedrovanzella.com}

\makeatother

\usepackage{babel}
\usepackage{listings}
\lstset {
    mathescape,
    frame=none
}
\renewcommand{\lstlistingname}{Listagem}

\begin{document}

\maketitle
\begin{abstract}
\end{abstract}
\begin{resumo}
\end{resumo}


\section{Introdução}\label{section:intro}

São chamados Números Escadinha aqueles que seguem um conjunto simples
de regras:
\begin{lyxlist}{00.00.0000}
\item [{1.}] Não começam com o dígito zero
\item [{2.}] Não possuem dígitos repetidos
\item [{3.}] Entre um dígito e o seguinte, a diferença em módulo não é
superior a 2.
\end{lyxlist}
O problema a ser resolvido neste artigo é o de se encontrar quantos
Números Escadinha existem em uma dada base. Fica evidente que eles
são finitos, devido à Regra 2 - o número de dígitos é limitado ao
tamanho da base, senão dígitos começariam a se repetir.


\section{Primeira Tentativa}\label{section:primeira}


\subsection{Algorítmos}\label{section:primeira:algoritmos}


\subsubsection{Escadinha}\label{section:primeira:algoritmos:escadinha}

\begin{lstlisting}
escadinha(base):
  count $\leftarrow$ 0
  num $\leftarrow$ [0] 
	
  enquanto num $\neq$ maxnum: 
    num $\leftarrow$ incrementa(num, base) 
    se valido(num):
      count++

  $\rightarrow$ count
\end{lstlisting}

Onde {\sf [0]} é um {\em array} contendo o inteiro $0$.



\subsection{Resultados}\label{section:primeira:resultados}
Obviamente o algoritmo é ruim. Mas quão ruim? Seria possível resolver o problema para uma entrada razoavel? A fim de curiosidade, o algorítmo foi implementado e os resultados podem ser vistos na Tabela~\ref{table:resultados-1}.

\begin{table}[h]
\caption{Primeira Tentativa}
\label{table:resultados-1}
\begin{tabular}{ll}
  {\sf num} & Tempo (s) \\
  \hline
  2 & 0.01 \\
  3 & 0.01 \\
  4 & 0.01 \\
  5 & 0.02 \\
  6 & 0.09 \\
  7 & 1.25 \\
  8 & 23.9 \\
  9 & 540.9    
\end{tabular}
\end{table}



\end{document}
